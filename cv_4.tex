%%%%%%%%%%%%%%%%%%%%%%%%%%%%%%%%%%%%%%%%%
% Medium Length Professional CV
% LaTeX Template
% Version 2.0 (8/5/13)
%
% This template has been downloaded from:
% http://www.LaTeXTemplates.com
%
% Original author:
% Trey Hunner (http://www.treyhunner.com/)
%
% Important note:
% This template requires the resume.cls file to be in the same directory as the
% .tex file. The resume.cls file provides the resume style used for structuring the
% document.
%
%%%%%%%%%%%%%%%%%%%%%%%%%%%%%%%%%%%%%%%%%

%----------------------------------------------------------------------------------------
%	PACKAGES AND OTHER DOCUMENT CONFIGURATIONS
%----------------------------------------------------------------------------------------

\documentclass{resume} % Use the custom resume.cls style

\usepackage[left=0.75in,top=0.6in,right=0.75in,bottom=0.6in]{geometry} % Document margins
\usepackage{enumitem}
\usepackage{hyperref}
\setlist[itemize]{noitemsep, topsep=0pt}
\name{Siddhartha Thota} % Your name
% \address{56 Yarmouth Road, Toronto} % Your address
\address{Machine learning engineer, passionate programmer }
\address{\href{http://www.cs.toronto.edu/~thota}{www.cs.toronto.edu/$\mathtt{\sim}$thota} \\ (416) 666 7834 \\ \href{mailto:thota@cs.toronto.edu}{thota@cs.toronto.edu} } % Your phone number and email

\begin{document}

%----------------------------------------------------------------------------------------
%	EDUCATION SECTION
%----------------------------------------------------------------------------------------

\begin{rSection}{Education}

{\bf University of Toronto, Canada} \hfill {\em Sept 2017 - Dec 2018} \\ 
MSc. in Applied Computing, Current GPA 3.85 \\
Courses: \smallskip \\
  \begin{tabular}{ l l }
  Machine Learning and Data Mining & Topics in Cloud, Mobile and Pervasive Computing \\
  Machine Learning in Computer Vision & Topics in Ubiquitous Comptuing: Assistive Technology
  \end{tabular} \smallskip \\
Research projects:
  \begin{itemize}
    \item Sidewalk terrain estimation by clustering sensor data to assist wheelchair users
    \item Scheduling algorithms for applications on heterogenous multi-tier clouds
  \end{itemize}


{\bf National Institute of Technology Karnataka, India} \hfill {\em July 2010 - March 2014} \\ 
BTech. in Information Technology, GPA 8.47 \\
Research project:
  \begin{itemize}
    \item  Identifying ragas (musical modes in Indian classical music) using HMMs
  \end{itemize}


\end{rSection}

%----------------------------------------------------------------------------------------
%	WORK EXPERIENCE SECTION
%----------------------------------------------------------------------------------------

\begin{rSection}{Experience}

\begin{rSubsection}{Adobe India, Bangalore}{\em June 2014 - Aug 2017}{}{}
\textit{Member of Technical Staff - II} 
\item Designed and built a multi-cloud service management framework on Azure and AWS.
\item Developed a Spark MLLib-based framework to generalize ML algorithms.
\item Built data pipelines, configuration services and scheduling interfaces for setting up model training.
\item Collaborated with data scientists and algorithm engineers, delivered deployments in production. \smallskip \\
\textit{Member of Technical Staff} 
\item Optimized customer spends using a customized analytical marketing mix model solver.
\item Built massive ETL pipelines, provided seasonality and multiple campaign support to customers.
\item Gained experience of working in a highly agile, self-motivated research project.
\end{rSubsection}

\end{rSection}

%----------------------------------------------------------------------------------------
%	TECHNICAL STRENGTHS SECTION
%----------------------------------------------------------------------------------------

\begin{rSection}{Technical Strengths}

\begin{tabular}{ @{} >{\bfseries}l @{\hspace{6ex}} l }
Computer Languages & Python, JS, Java, Scala, C/C++, R, Octave, MATLAB, HTML/CSS \\
ML \& Data Science & scikit-learn, Spark-MLLib, Pandas, TensorFlow \\
Distributed Computing & Spark, Docker, AWS/Azure, Mesos/Marathon etc. \\
Tools & Vim, git, linux, shell \\
\end{tabular}

\end{rSection}

\begin{rSection}{Awards and Publications}
  \begin{itemize}
    \item Best poster award, CMU Winter School, Bangalore, 2012: Raga detection in Indian Classical Music using HMMs
    \item Conference paper: Scale independent raga identification using chromagram patterns and swara based features (IEEE Explore) - IEEE, ICMEW, 2013
  \end{itemize}
\end{rSection}

\end{document}
